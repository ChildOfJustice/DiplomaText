\prettypart{ВВЕДЕНИЕ}

При работе любого предприятия, создании продуктов для конечных потребителей, предоставлении услуг клиентам, генерируется колоссальное количество разнотипных и разнородных данных –- изображения, видео, аудиозаписи, бинарные файлы, иногда даже структурированные файлы, что приводит к невозможности эффективного хранения этой информации в реляционных хранилищах данных.

В качестве типичного примера можно рассмотреть систему версионирования файлов, обработки метаданных, отслеживания прав доступа в корпоративной сети компании. Все данные должны храниться требуемым образом и быть организованы так, чтобы можно быстро исполнить запрос на поиск в любой момент времени.

В современном мире таким задачам уделено огромное внимание  и введено специальное название подобных платформ: Digital Asset Management Software или Системы Управления Цифровыми Активами (далее DAM-системы).

\section*{\textbf{Сформированный рынок решений}}
Во многих российских компаниях до сих пор используют папки на внешних жёстких дисках или обычное облачное хранилище и для поддержки такого способа организации контента требуются отдельные сотрудники, которые вручную решают задачи поиска нужных файлов и следят за их структурой и целостностью.
Более того, при использовании облачных хранилищ всегда имеются ограничения на максимальный размер, количество скачиваний в сутки, при экспорте невозможно управлять метаданными. Нет возможности создавать группы пользователей с разными правами и отслеживать процесс добавления и хранения файлов, а также затруднителен процесс ведения лога всех действий.

\section*{Проблемы существующих DAM"=систем}
Проблема выбора и правильного использования Digital Asset Management систем в последние годы стала довольно серьёзной. Решения на европейском рынке готовы предложить свою DAM-систему, с внедрением в окружение компании, настройкой и дальнейшей поддержкой. Но есть несколько важных критериев, которые нельзя проигнорировать при поиске подходящего решения для управления цифровыми активами организации.

Один из них – использование новейших технологий, доступных в облачном окружении. Перенести изначальные средства администрирования файлов из дата центров компании в облако не является облачным решением. Только используя контейнеризацию, бессерверный подход к созданию архитектуры и микросервисы, можно полностью воспользоваться потенциалом облака.

Второй критерий – это доступность и простота интеграции в уже имеющиеся ресурсы компании. Огромное количество DAM-систем предлагают в своих услугах настройку, конфигурирование своего решения или требуют огромную цену при использовании, до 50\$ в день. Большие организации без каких-либо проблем могут позволить себе дорогое подключение DAM-системы к своему окружению. Но для менее крупных компаний это может оказаться слишком затратно и по количеству средств, и по вложению сотрудников при ручной интеграции или при обучении пользованию мощной DAM-системой.

В данной работе будут рассмотрены следующие аспекты: что такое Digital Asset Management система, способы создания DAM"=систем, преимущества облачных решений, а также будет поставлена и выполнена задача создания своей системы управления цифровыми активами с учетом всех проведенных исследований в выбранной области.
