% \clearpage
% \chapter{ИСПОЛЬЗУЕМЫЕ ТЕХНОЛОГИИ}

\section{Описание всех использованных технологий, инструментов и подходов}

\subsection{AWS}
Amazon Web Services (AWS) – это самая распространенная в мире облачная платформа с широчайшими возможностями, предоставляющая более 200 полнофункциональных сервисов для центров обработки данных по всей планете.

Использование AWS при создании DAM"=системы занимает основную ключевую роль. Это размещение пользовательского интерфейса с предоставлением возможностей регистрации клиентов, обработка всех запросов, хранение загружаемых в Digital Asset Management систему файлов, не реляционная база данных для управления всей информацией (о файлах, правах и создаваемых пользователями папках). Все эти сервисы интегрированы между собой.

\subsection{Serverless app}
Serverless – это подход к созданию архитектуры, в которой используются облачные сервисы, полностью управляемые cloud-провайдером, в данном случае Amazon Web Services. 
 Пример: AWS Lambda - это управляемый событиями сервис, который позволяет запускать код в ответ на запросы из более чем 150 встроенных источников AWS - и все это без управления какими-либо серверами.

Бессерверный подход является основным фундаментом данной DAM"=системы --- вся работа с файлами и правами происходит через несколько Lambda"=функций.
Все запросы изначально проходят через AWS API Gateway, этот сервис распределяет нагрузку управления системой с помощью REST методов.
Далее, Lambda"=функция извлекает необходимую информацию из тела запроса или из query параметров и выполняет указанную задачу.
При этом данный сервис способен работать со многими другими AWS сервисами. В данном случае --- с DynamoDB table.
Таким образом, добавление новых записей в базу данных, их изменение и поиск осуществляется полностью без какого-либо арендованного и постоянно работающего сервера.
Только лишь когда приходит непосредственный запрос на API, облако AWS запустит соответствующую Lambda"=функцию, она совершит все необходимые действия и вернет результат.
При этом оплата взимается за время работы функции и затраченную память

\subsection{DynamoDB (NoSQL database)}
Amazon DynamoDB – это база данных пар «ключ‑значение», которая обеспечивает задержку менее 10 миллисекунд при работе в любом масштабе. Используя правильный подход к моделированию хранимых данных, в одной таблице можно разместить несколько видов различных записей, что даст возможность проводить не слишком сложные JOIN"=запросы и агрегировать структуру в необходимом приложению виде.

В разработанном программном продукте этот сервис играет роль хранилища информации о всех клиентах, созданных ими кластерах (smart-папках) и метаданных о загруженных в DAM"=систему файлах. Благодаря Global Secondary Index (GSI) появляется возможность делать запросы в базу данных с несложной агрегацией, например: получить все кластеры для указанного пользователя (Задан идентификатор пользователя - Hash Key, нужно найти все кластеры, id владельца которых равен указанному id пользователя).


\subsection{Deployment (CloudFormation template)}
Сервис AWS CloudFormation позволяет разработчикам и компаниям без труда создавать наборы связанных ресурсов AWS и сторонних библиотек, обеспечивать их упорядоченное и предсказуемое выделение, а также управление ими и всё это поддерживается как код. Это позволяет сделать контролирование версий архитектуры и откат к рабочему состоянию окружения в случае ошибки при обновлении. В дальнейшем планируется использовать Terraform – этот инструмент позволяет применить декларированный подход к управлению ресурсами в облаке. Указывается желаемое состояние, а ядро Terraform само вычислит, какие изменения в инфраструктуре следует применить, исходя из начального состояния.

\subsection{React}
React - это бесплатная библиотека JavaScript для разработки пользовательских интерфейсов или его компонентов. React поддерживается Facebook и сообществом отдельных разработчиков. В данной DAM"=системе эта библиотека используется как база для создания всех частей, связанных с пользовательским взаимодействием с приложением. Таблицы, кнопки и все поля настроек системы управления цифровыми активами выполнены как одностраничное приложение (Single Page Application), написанное на языке TypeScript, что позволяет избежать неприятных ошибок и недопониманий типизации при разработке. 

\subsection{API}
Application Programming Interface - это программная часть, позволяющая двум приложениям взаимодействовать друг с другом. В разработанной DAM"=системе, API позволяет при получении любого события со стороны пользователя, послать этот запрос на обработку с дальнейшим изменением данных, таких как добавление нового кластера, загрузка файла в систему управления цифровыми активами, изменение или создание прав на пользование smart"=папкой и удаление перечисленных сущностей. Благодаря сервису AWS API Gateway происходит распределение нагрузки, и поддерживается стиль архитектуры RESTfull API. После создания запроса в интерфейсной части приложения, он отправляется на endpoint, предоставляемый API Gateway, далее запускается Lambda"=функция, ответственная за обработку конкретного запроса. Это позволяет обеспечить архитектурный подход Serverless, так как AWS Lambda - это полностью управляемый облачным провайдером сервис, который может максимально быстро масштабироваться, а оплата происходит только лишь за количество времени и памяти, которые были нужны при каждом запуске такой Lambda"=функции. В данном проекте использовался язык Python для работы с базой данных при поступлении события от пользователя на Lambda.


\subsection{JWT для авторизации и контроля сессий}
JWT - это открытый стандарт (RFC 7519), который определяет компактный и автономный способ безопасной передачи информации между сторонами в виде Java Script Notation Object. Эту информацию можно проверить на подлинность её отправителя, потому что она имеет цифровую подпись. JWT могут быть подписаны с использованием алгоритма HMAC или пары открытого / закрытого ключей с использованием RSA или ECDSA. В созданном приложении этот стандарт используется для авторизации пользователей DAM"=системы, а также для управления временем, через которое пользователь будет обязан авторизоваться в системе со своим логином и паролем снова. Сервис AWS API Gateway поддерживает автоматическую проверку JWT токенов на подлинность, при этом, токены предоставляет сервис AWS Cognito, также ответственный за добавление новых пользователей в систему, за проверку введенных ими при регистрации данных и за их менеджмент.

\subsection{DevOps}
DevOps методология - это сочетание подходов, практик и инструментов, которые повышают способность организации предоставлять приложения и услуги с высокой скоростью: разрабатывать и улучшать продукты более быстрыми темпами, чем организации, использующие традиционные процессы разработки программного обеспечения и управления инфраструктурой.

В рамках модели DevOps группы разработчиков и эксплуатации (QA) больше не разделены. Эти две группы вместе работают над проектом на протяжении всего цикла разработки приложения - создание новых компонентов, тестирование, развертывание и эксплуатация. После мониторинга готового продукта выдвигаются новые предложение о его улучшении и цепочка повторяется снова.

Эти группы используют методы для автоматизации процессов, которые исторически выполнялись вручную и выполнялись медленно. Использование стека технологий и инструментов помогают быстро и надежно управлять и развивать приложения. Эти инструменты также помогают инженерам независимо выполнять задачи, которые обычно требовали бы помощи других команд(например, развертывание кода или обеспечение инфраструктуры для разработки и тестирования приложения), что еще больше увеличивает скорость работы команды.

В данном приложении этот подход использовался с целью полной автоматизации развертывания DAM"=системы в облаке AWS. Созданные программные части на языке Pyhton и Shell script позволяют всего одним нажатием кнопки запустить полностью автономный процесс подготовки всех Amazon сервисов, компиляции кода пользовательского интерфейса и подключения всех частей программы друг к другу. Более подробно об этом будет рассказано в следующей главе.

% OPTIONAL:
% CloudWatch
% CloudTrail