\clearpage
\chapter{Теоретическая часть}

% \section{Теоретическая часть}
\section{Исследование решений, существующих в области выбранной темы проекта}
Основными объектами исследования являются устройство, назначение, применение и способы создания систем управления цифровыми активами (DAM-систем). Также, при рассмотрении конкретных выбранных технологий для увеличения эффективности такой системы, были затронуты следующие аспекты: 
\begin{enumerate}
  \item aрхитектурный подход Serverless, разработка DAM"=систем, реализующих этот подход, а также способы создания Serverless приложений в облаке AWS; 
  \item использование NoSQL баз данных для управления метаданными;
  \item правильный подход к управлению средствами, на что именно и как тратится бюджет компании при использовании облачных провайдеров;
  \item использование машинного обучения для определения метаданных в DAM"=системах.
\end{enumerate}

\subsection{DAM"=системы как важный элемент предприятия и варианты их внедрения}
В публикации \cite{DamImpl} рассматривается роль использования DAM-систем и почему это важно для компаний и бизнеса.

Более подробное описание что такое Digital Asset Management System можно посмотреть на веб-ресурсе \cite{whatIsDamSys}, который преподносит не только определение DAM-систем, но и что именно они позволяют достичь и какие усовершенствования они вносят в бизнес компании \cite{DigitalAssetManagement}.

Сергей Фомин, основатель компании Picvario, которая создала одну из первых DAM-систем в России, в своей статье \cite{SergeyPhomin} досконально показывает недостатки использования обычных дисков или облачных хранилищ в целях использования организацией.

Anna Jennifer в своей публикации \cite{HowToImplementDam} рассматривает пример внедрения системы управления цифровыми активами в организацию для обработки фотографий и видео. Анализ возможностей DAM-систем и их применение в реальном бизнесе \cite{DamBusinessIntegration}. 

\subsection{Разработка DAM"=систем}

Разработка такой системы включает в себя оптимальное использование правильно подобранной базы данных или кластера из нескольких, возможность гладкой интеграции в уже созданные структуры и, конечно же, хорошо продуманные способы работы с метаданными, удобный поиск, интерфейс для пользователя или API для подключения через команды или протоколы. Toni Ahonen в своей книге \cite{DamDevelopment} про создание DAM-систем разобрал все основополагающие моменты.
Не стоит забывать про стоимость использования облачных технологий. Правильный подход к управлению средствами очень важен, так как при колоссальных возможностях cloud-провайдеров есть вероятность не заметить на что именно тратится бюджет компании \cite{AwsCostsManagement}.

После изучения способов и проблем разработки систем управления цифровыми активами, было выявлено два ключевых аспекта, использование которых может увеличить актуальность и оригинальность выпускной квалификационной работы.

Первый из них был разработан Austen Collins, архитектурный подход Serverless \cite{AwsServerless}. В статье \cite{ProdServerlessEnv}, Lee Hyungro рассмотрел основные преимущества и реализацию Serverless Computing в AWS облаке, принцип работы основного сервиса Lambda и для каких приложений serverless может в разы усовершенствовать принципы работы и поддержки.

В книге Serverless Computing: Current Trends and Open Problems \cite{ServerlessTrendsAndProblems} проведён глубокий анализ Serverless вычислений и новых возможностей, которые появляются вместе с проблемами правильной реализации данного метода создания структуры проектов.

Популярность и действенность данного архитектурного подхода подтверждают \cite{ServerlessFrameworks} многочисленные разработанные библиотеки и фреймворки для программистов, которые работают с Serverless приложениями.

Второй аспект – использование NoSQL \cite{NoSqlDesignEvolution} базы данных для работы с метаданными, правами на файлы и другими записями в DAM-системе \cite{NoSQLFramework}. Изучив все сильные стороны не реляционных баз данных, была рассмотрена подробная информация о создании дизайна и структуры таких хранилищ \cite{NoSqlDesign}. Основной базой данных была выбрана AWS DynamoDB \cite{AwsDynamoDB}. Также был изучен курс о разработке моделей данных и настройке этого сервиса \cite{CloudGuruCourses}. Важно заметить, что эта NoSQL база данных использует специально разработанную хеш-функцию для размещения информации по partitions, чтобы увеличить быстродействие и возможности для масштабирования \cite{NoSqlDataPartition}.
Мощность не реляционных баз данных при использовании в крупных организациях сложно переоценить \cite{NoSqlForEnterprise}.

В настоящее время всё больше программных продуктов начинает использовать машинное обучение, что, конечно же, касается и Digital Asset Management систем \cite{DamAndAi}.

\subsection{Результаты проведённых исследований}
На основании исследования вышеуказанных источников, был сделан вывод, что DAM-систем, использующих новую технологию Serverless \cite{MicrosoftServerlessOnAws} очень мало или почти нет, в основном это либо программное обеспечение, которое устанавливается на отдельный компьютер и работает с базой данных. Либо это облачная структура, которая разворачивается на предоставляемых компьютерах (серверах) в облаке \cite{AwsCloud} и связывается с помощью протоколов с уже работающей архитектурой, что не является Serverless приложением. 

Далее перечислены сервисы, используемые как основные компоненты при создании такого программного продукта:
\begin{enumerate}
\item AWS Lambda \cite{AwsLambda} - это управляемый событиями сервис вычислений, который позволяет запускать код в ответ на события из более чем 150 встроенных источников AWS, и все это без управления какими-либо серверами \cite{ServerlessNodeJs};

\item AWS S3 \cite{AwsS3} - Object-storage для хранения файлов любых размеров;

\item amazon API Gateway V2.0 \cite{AwsGateway} - Сервис для обработки и перенаправления REST-запросов.
\end{enumerate}
В дополнение ко всему этому было детально изучено устройство защиты информации в облаке \cite{CloudSecurity}.


\section{Постановка задачи, решённой в ВКР}

Учитывая описанные недостатки, в данной работе была поставлена задача реализации усовершенствованной DAM-Системы, используя Serverless технологии микросервисов AWS.

Упомянутая система управления цифровыми активами позволит перед добавлением нового файла заполнить или отредактировать поля метаданных, организовать максимально оптимизированный поиск файлов, причём его всегда будет возможно полностью настроить под конкретную задачу. Благодаря хранению метаданных отдельно от самого файла, нет необходимости скачивать каждый раз сам объект, что может очень сильно повысить скорость изменения его описания и свойств, избегая при этом получения избыточной информации.

При применении данного подхода, удается снять ограничения на размер загружаемых и скачиваемых файлов, предоставляется способ манипулирования  правами на хранимые объекты, повышается уровень защиты данных, появляется возможность логирования истории операций и осуществления поддержки большого количества форматов метаданных. Также значительно упрощается интеграция в уже имеющиеся ресурсы компании благодаря предоставлению API через такой AWS сервис как API Gateway.

Ниже перечислены все последние технологии облака AWS (Amazon Web Services), которые используются в предлагаемом способе создания DAM-системы:
\begin{enumerate}
  \item подход к созданию Serverless архитектуры;
  \item использование одной из самых мощных NoSQL баз данных, позволяющую работать даже с взаимосвязанными (relational) данными и небольшими JOIN-запросами благодаря Global Secondary Indexes \cite{AwsDynamoDBGsi};
  \item защита, предоставляемая компанией Amazon для всех пользователей AWS \cite{AwsSecurity}.
\end{enumerate}


В результате проведённого в рамках выбранной темы исследования существующих решений, относящихся к поставленной задаче, были окончательно сформулированы объект, предмет и цель работы.

\textit{Объект работы} --- облачное решение для создания Системы Управления Цифровыми активами.

\textit{Предмет работы} --- использование такого архитектурного подхода как Serverless, а также Amazon Web Services для реализации DAM"=системы с автоматизированным процессом разворачивания в облаке.

\textit{Цель работы} --- разработка программного продукта для управления цифровыми активами организации с отличительными особенностями, которые перечислены далее:
\begin{enumerate}
\item {cloud-native} --- использование последних доступных технологий для увеличения эффективности и скорости работы продукта;

\item {serverless} --- Бессерверная архитектура позволяет платить лишь за реально используемые ресурсы при выполнении задач и обработке запросов пользователей;

\item {automated deployment to AWS cloud} --- быстрое разворачивание всей инфраструктуры, требуемой для работы DAM"=системы в один клик мыши. 
\end{enumerate}

Более того, если организация уже использует AWS сервисы для ведения своего бизнеса, то интеграция данной DAM-системы в окружение компании будет гораздо проще, например, есть возможность сразу же использовать API DAM"=системы через AWS API Gateway и Lambda функции, что существенно уменьшает затраты времени на подключение системы управления цифровыми активами к уже имеющимся готовым решениям предприятия.

Комплексное использование перечисленных элементов создает определенные преимущества в разработанной системе управления цифровыми активами.


После успешного создания демонстрационного приложения, было подготовлено и выполнено выступление на конференции “Гагаринские чтения” в 2021 году  \cite{DamAwsAnreyYuldashev}.
