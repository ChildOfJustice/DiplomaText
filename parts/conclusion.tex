\prettypart{ЗАКЛЮЧЕНИЕ}
% \section{Положения, выносимые на защиту}
% \section{Результаты работы}


В работе были рассмотрены и исследованы различные подходы к созданию DAM"=систем. Практическим результатом работы является разработанная система управления цифровыми активами, реализованная с помощью новейших облачных технологий. 

Основным преимуществом этой DAM-системы является скорость разворачивания в AWS окружении и простота подключения, использования. Права пользователей хранятся в базе данных, а не в AWS IAM, что упрощает архитектуру в облаке и не дает информации о доступе к файлам выходить за пределы Digital Asset Management системы. 

Это реализованная cloud-native DAM-система, для работы которой необходим только аккаунт AWS, все расходы можно увидеть через свой AWS счёт. Оплата происходит только за реально используемые ресурсы и количество запросов, как при любом другом использовании Serverless архитектуры. Работа с метаданными обеспечивается с помощью такой базы данных как AWS DynamoDB и происходит через REST API, запросы к которому обрабатываются AWS Lambda функциями. Файлы оптимально размещены и структурированы с помощью кластеров (smart-папок), права на доступ к каждой папке могут быть выданы ровно в необходимом для работы количестве. Ограничений на загрузку и скачивание файлов нет, так как AWS управляет своими сервисами и масштабирует их по мере необходимости.
Благодаря AWS CloudFormation и логически разделенным AWS Lambda функциям есть возможность удобного способа поддержки и добавления нового функционала в данную DAM-систему.

Сформулированная задача была полностью решена в рамках выпускной квалификационной работы. Разработанный программный продукт возможно развивать и дальше, добавляя новый функционал и используя огромные возможности облака AWS, такие как мониторинг всех процессов системы, машинное обучение для добавления метаданных, различные улучшения производительности и доступности на региональном уровне, а также дополнительные возможности защиты приложения.