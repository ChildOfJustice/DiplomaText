% \addcontentsline{toc}{prettypart}{СПИСОК ТЕРМИНОВ И СОКРАЩЕНИЙ}
\prettypart{ТЕРМИНЫ И ОПРЕДЕЛЕНИЯ}

% \newcolumntype{s}{>{\hsize=.25\hsize \raggedright\arraybackslash}X} \newcolumntype{b}{>{\hsize=.75\hsize}X}

% \begin{tabularx}{\textwidth}{|sb|} { 
%      >{\raggedright\arraybackslash}X 
%      >{\raggedright\arraybackslash}X 
%      >{\raggedright\arraybackslash}X  
%   }

%  API & --- & это набор определений и протоколов для создания и интеграции прикладного программного обеспечения. Иногда это называют контрактом между поставщиком информации и пользователем информации, устанавливающим контент, требуемый от потребителя (запрос), и контент, требуемый производителем (ответ). Например, дизайн API для службы погоды может указывать, что пользователь предоставляет почтовый индекс, а производитель отвечает протоколом, состоящим из двух частей, первая из которых соответствует высокой температуре, а вторая - низкой \\
%  item 21  & item 22  & item 23  \\
% \end{tabularx}


% \noindent
% \begin{tabular}{
%   p{\dimexpr.2\linewidth-2\tabcolsep-1.3333\arrayrulewidth}% column 1
%   p{\dimexpr.1\linewidth-2\tabcolsep-1.3333\arrayrulewidth}% column 2
%   p{\dimexpr.25\linewidth-2\tabcolsep-1.3333\arrayrulewidth}% column 3
%   }
%   \raggedright TERM& --- & asdgasgasgasfgadfgafadf \\
% \end{tabular}





\RaggedRight \noindent \justify \sloppy API
\begin{minipage}[t][][t]{3.15cm}
\RaggedLeft
  ---
\end{minipage}
\begin{minipage}[t][][t]{0.75\textwidth}
\RaggedRight  \justify \sloppy
 это набор определений и протоколов для создания и интеграции прикладного программного обеспечения. Иногда это называют контрактом между поставщиком информации и пользователем информации, устанавливающим контент, требуемый от потребителя (запрос), и контент, требуемый производителем (ответ). Например, дизайн API для службы погоды может указывать, что пользователь предоставляет почтовый индекс, а производитель отвечает протоколом, состоящим из двух частей, первая из которых соответствует высокой температуре, а вторая - низкой \\
\end{minipage}



\RaggedRight \noindent \justify \sloppy Back"=End
\begin{minipage}[t][][t]{1.9cm}
\RaggedLeft
  ---
\end{minipage}
\begin{minipage}[t][][t]{0.75\textwidth}
\RaggedRight  \justify \sloppy
 это серверная часть, та часть приложения, которая скрыта от пользователя. Эта часть отвечает за обработку данных, их хранение и математические операции \\
\end{minipage}


% \begin{minipage}{4cm}
\RaggedRight \noindent \justify \sloppy DAM"=система
% \end{minipage}%
\begin{minipage}[t][][t]{1cm}
\RaggedLeft
  ---
% 123.456.7890
\end{minipage}%
\begin{minipage}[t][][t]{0.75\textwidth}
\RaggedRight  \justify \sloppy
 это программный продукт, предназначенный для управления информацией, используемой в бизнес"=целях. Сортировка, поиск необходимых файлов и структуризация для быстрого и удобного хранения данных, а также менеджмент прав на использование всей этой информацией сотрудниками компании. (DAM"= сокращение от Digital Asset Management) \\
\end{minipage}



\RaggedRight \noindent \justify \sloppy Endpoint
\begin{minipage}[t][][t]{1.9cm}
\RaggedLeft
  ---
\end{minipage}
\begin{minipage}[t][][t]{0.75\textwidth}
\RaggedRight  \justify \sloppy
это одна из рубежных точек канала связи. Когда две системы связываются друг с другом через API, точки взаимодействия и называются endpoints. Для API конечная точка может включать URL-адрес сервера или службы. Каждая конечная точка - это место, из которого API-интерфейсы могут получить доступ к ресурсам, необходимым для выполнения своей функции \\
\end{minipage}



\RaggedRight \noindent \justify \sloppy Front"=End
\begin{minipage}[t][][t]{1.7cm}
\RaggedLeft
  ---
\end{minipage}
\begin{minipage}[t][][t]{0.75\textwidth}
\RaggedRight  \justify \sloppy
это то, что пользователь видит и с чем взаимодействует (пользовательский интерфейс) \\
\end{minipage}



\RaggedRight \noindent GSI и LSI
\begin{minipage}[t][][t]{1.7cm}
\RaggedLeft
  ---
\end{minipage}
\begin{minipage}[t][][t]{0.75\textwidth}
\RaggedRight  \justify \sloppy
это вторичные индексы базы данных AWS DynamoDB. GSI (Global Secondary Index) позволяет выполнять запросы по всей таблице и по всем разделам, меняя основной ключ по которому будет производиться агрегация данных. LSI (Local Secondary Index) позволяет выполнять запросы по одной секции, указанной значением Sort Key в запросе, LSI может быть применен только для изменения вторичного ключа, по которому производиться отбор и сортировка данных, основной ключ поиска (Primary Key) будет совпадать с Primary ключом таблицы \\
\end{minipage}



\RaggedRight \noindent \justify \sloppy Query parameters
\begin{minipage}[t][][t]{0.4cm}
\RaggedLeft
  ---
\end{minipage}
\begin{minipage}[t][][t]{0.75\textwidth}
\RaggedRight  \justify \sloppy
это параметры запроса, определенный набор параметров, прикрепленный к концу строки URL-адреса. Они являются расширениями URL-адреса, которые используются для определения содержимого или действий на основе передаваемых данных. Чтобы указать параметры запроса в конец URL-адреса добавляется знак «?», за которым следует query параметр \\
\end{minipage}



\RaggedRight \noindent \justify \sloppy RESTfull API
\begin{minipage}[t][][t]{1.3cm}
\RaggedLeft
  ---
\end{minipage}
\begin{minipage}[t][][t]{0.75\textwidth}
\RaggedRight  \justify \sloppy
REST API (также известный как RESTful API) - это интерфейс прикладного программирования (API или веб-API), который соответствует ограничениям архитектурного стиля REST и позволяет взаимодействовать с веб-службами RESTful. REST означает передачу репрезентативного состояния и был создан компьютерным ученым Роем Филдингом \\
\end{minipage}
\RaggedRight \noindent \justify \sloppy Smart-папка
\begin{minipage}[t][][t]{1.4cm}
\RaggedLeft
  ---
\end{minipage}
\begin{minipage}[t][][t]{0.75\textwidth}
\RaggedRight  \justify \sloppy
это способ группировки файлов из разных мест в одну папку, по определенным критериям или вручную. При этом при размещении одного файла в несколько таких папок не создаются копии, а лишь добавляются указатели на этот файл в список каждой отдельной smart-папки. Отображение содержимого осуществляется с помощью обработки запроса к базе данных. В данной работе термин "кластер" имеет то же значение
\end{minipage}